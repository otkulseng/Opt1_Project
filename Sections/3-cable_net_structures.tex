\section{Cable net structures}
In this section, we are analysing the situation where all members of the structure are cables, and where we fix certain nodes in order to ensure a solution exists.

This gives us the following optimization problem:

\begin{equation}
    \underset{X}{\text{min }} E(X) = \sumset{E} \ece + \ee \quad \text{s.t. } x^{(i)} = p^{(i)}, i = 1,...,M
    \label{cableNet}
\end{equation}

This optimization problem is $C^1$. We have already shown that a more general problem is continious, therefore \eqref{cableNet} is continious. It's clear that $\ee = \sum_{i=1}^N m_i g x_3^{(i)} \in C^{\infty}$, so the limitation is equation \eqref{cableElast}.

As $\ece$ is a piecewise function, we need to ensure that the derivative at $\xnorm = \el$ is equal to zero.

\begin{equation}
    \at{\nabla \ece}{\xnorm = \el} = \at{ \frac{k}{\el^2}(\xnorm - \el) \nabla (\xnorm-\el) }{\xnorm = \el} = 0
\end{equation}

The last equality is due to the fact that

\begin{equation}
    \nabla (\xnorm - \el) = \nabla \sqrt{(x^{(i)} -x^{(j)})^2} = \frac{2(x^{(i)} - x^{(j)})}{2 \sqrt{(x^{(i)} -x^{(j)})^2} } = \frac{(x^{(i)} - x^{(j)})}{\xnorm}
\end{equation}
which shows that it is $C^1$.
\textbf{kommentar: Den notasjonen føltes veldig ulovlig}

Note that the problem is typically not $C^2$ because that would require that the derivative above is continious, which it clearly is not unless
$\sumset{E} \ece = 0$, which is only possible if we have $0$ cables or cables with $0$ energy, and these are not particularly interesting examples..