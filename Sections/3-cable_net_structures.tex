\section{Cable net structures}
In this section, we are analysing the situation where all members of the structure are cables, and where we fix certain nodes in order to ensure that a solution exists. This gives us the following optimization problem:
\begin{equation}
    \underset{X}{\text{min }} E(X) = \sumset{E} \ece + \ee \quad \text{s.t. } x^{(i)} = p^{(i)}, i = 1,...,M
    \label{cableNet}
\end{equation}
In order to solve this optimization problem, we first have to show some properties about the function. We have already shown that the more general problem \eqref{totalEnergy} is continious, therefore \eqref{cableNet} is continious.

\textbf{Theorem: The function given in \eqref{cableNet} is $C^1$.}

We will look at this term by term
The gradient of $\ee$ is 
\begin{equation}
    \nabla E_{\textbf{ext}} = \nabla \sum_{i=1}^N m_i g x_3^{(i)}
    = (0,0,m_1 g, 0,0,m_2 g,...,0,0,m_N g)
    \label{gradient_external_force}
\end{equation}
which is continuous. In fact, it's clear that $\ee \in C^{\infty}$.

$\ece$ is obviously differentiable when $\xnorm \neq \el$. If $\ece$ is to be different from zero in it's entire domain we need 
\begin{equation}
    \underset{\xnorm \to \el ^+}{\text{lim}} \nabla \ece = 0    
\end{equation}
We will calculate the partial derivative with respect to $x_s^{(i)}$ where $s$ represents one of the three directions, that is: $s=1,2,3$. As we take the limit from above, we only consider the expression of $\ece$ when $\xnorm > \el$.
\begin{align}
    \delx \ece = &\delx  \frac{k}{2 \el^2} (\xnorm - \el)^2\\ 
    &= \delx\frac{k}{\el^2}(\xnorm - \el) \delx \bigg(\sqrt{ \sum_{q=1}^3 (x_q^{(i)}-x_q^{(j)})^2}-\el\bigg)\\&=\frac{k}{\el^2} (\xnorm - \el) \frac{1}{2\sqrt{\sum_{q=1}^3 (x_q^{(i)}-x_q^{(j)})^2}}\delx \sum_{q=1}^3(x_q^{(i)}-x_q^{(j)})^2\\ &= \frac{k}{\el^2} (1 - \frac{\el}{\xnorm}) (x_p^{(i)} - x_p^{(j)}) \delx (x_p^{(i)} - x_p^{(j)})\\ &= \frac{k}{\el^2} (1- \frac{\el}{\xnorm}) (x_p^{(i)} - x_p^{(j)})
\end{align}

Similarly, we have 
\begin{equation}
    \frac{\partial}{\partial x_p^{(j)}} \ece =  \frac{k}{\el^2} (1- \frac{\el}{\xnorm}) (x_p^{(j)} - x_p^{(i)})
\end{equation} where the only difference is a factor of $-1$.

We take the limit:
\begin{equation}
    \underset{\xnorm \to \el ^+}{\text{lim}} \delx \ece =    \underset{\xnorm \to \el ^+}{\text{lim}} \frac{k}{\el^2} (1- \frac{\el}{\xnorm}) (x_p^{(i)} - x_p^{(j)}) = 0
\end{equation}
This holds for all partial derivatives, which shows that the function is $C^1$. Note that $\xnorm>\el$, which means that we never divide by zero. Thus we have a sum of $C^1$ functions, which is $C^1$. However, the function is not $C^2$. We can show this by showing that the following second order partial derivative is not continuous when $\xnorm \to \el^+$:
\begin{align*}
    \frac{\partial^2}{\partial (x_p^{(i)})^2} \ece &= 
    \delx \frac{k}{\el^2} (1- \frac{\el}{\xnorm}) (x_p^{(i)} - x_p^{(j)}) \\&
    = \delx \frac{k}{\el^2} (1- \frac{\el}{\xnorm}) (x_p^{(i)}) \\&
    =\frac{k}{\el^2} - \frac{k}{\el} \delx \frac{x_p^{(i)}}{\xnorm} \\&
    =\frac{k}{\el^2} -\frac{k}{\el} \frac{\xnorm - x_p^{(i)} \bigg( \delx \sqrt{\sum_{q=1}^3 x_q^{(i)}-x_q^{(j)}} \bigg)}{\xnorm^2} \\&
    =\frac{k}{\el^2} -\frac{k}{\el} \frac{\xnorm - x_p^{(i)} \bigg( \frac{x_p^{(i)}-x_p^{(j)}}{\xnorm} \bigg)}{\xnorm^2} \\&
    =\frac{k}{\el^2} -\frac{k}{\el} \frac{\xnorm^2 - x_p^{(i)} \bigg( x_p^{(i)}-x_p^{(j)} \bigg)}{\xnorm^3}
\end{align*}
Now we take the limit:
\begin{align*}
    \underset{\xnorm \to \el ^+}{\text{lim}} %\frac{\partial^2}{\partial (x_p^{(i)})^2} \ece
    \frac{k}{\el^2} -\frac{k}{\el} \frac{\xnorm^2 - x_p^{(i)} \bigg( x_p^{(i)}-x_p^{(j)} \bigg)}{\xnorm^3}
    = \frac{k}{\el^2} -\frac{k}{\el} \frac{\el^2 - x_p^{(i)} \bigg( x_p^{(i)}-x_p^{(j)} \bigg)}{\el^3}
\end{align*}
This is only equal to $0$ if 
\begin{equation}
     \el^2 = \el^2 - x_p^{(i)}(x_p^{(i)}-x_p^{(j)}) \implies
     x_p^{(i)}(x_p^{(i)}-x_p^{(j)})=0
\end{equation}

Which does not hold in general. Therefore the function is not typically $C^2$. $\square$

Note that the problem is $C^2$ if we have
$\sumset{E} \ece = 0$, but that is only possible if we have $\mathcal{E} = \emptyset$, or if no cables are stretched. These are not particularly interesting situations.

\subsection{Convexity}
Convexity is another property that is of great importance when considering the choice of optimization algorithm. We will look at the convexity term by term.
$\ee =\sum_{i=1}^N m_i g x_3^{(i)}$ is convex, but not strictly convex:
\begin{equation*}
    E(\lambda X +(1-\lambda) Y) = \sum_{i=1}^N m_i g (\lambda x_3^{(i)} + (1-\lambda) y_3^{(i)})
    =\sum_{i=1}^N \lambda m_i g x_3^{(i)} + (1-\lambda) m_i g  y_3^{(i)} = \lambda E(X) + (1-\lambda) E(Y)
\end{equation*}
For the elastic energy of the cable, we have three cases. 

If $\mathcal{E} = \emptyset $ we have $\sumset{E} \ece = 0$ which is convex, but not strictly convex. 

If $\xnorm > \el \quad \forall \quad \el \in \mathcal{E}$, we have strict convexity. Proof:

\textbf{Hensiktsmessig å ha dette inni beviset?}
Note first that 
\begin{equation}
    \ece = E^{cable}_{elast}(L(\el)) = E^{cable}_{elast}(\xnorm,\el),
\end{equation} thus $\ece : \mathbb{R}^+ \times \mathbb{R}^+ \rightarrow \mathbb{R}$. Define 

\begin{equation}
    E_{ij}(y) = E(y,\el) 
    \begin{cases}
    \frac{k}{2 \el^2} (y-\el)^2 & \text{if} \quad y > \el \\
    0 & \text{if} \quad y \leq \el
    \end{cases}
\end{equation}

And observe that 
\begin{equation}
    \ece 
\end{equation}

,$\xnorm \leq \el$ the expression is $0$, which is convex, but not strictly convex. For $\xnorm > \el$
we will consider a given edge $e_{ij}$ where we assume that $\xnorm > \el$. Note that this assumes that the mass of node $i$ and $j$ are non-zero. Let $0 < \lambda < 1$, and define $p:=x^{(i)} - x^{(j)}, \tilde{p} = \tilde{x}^{(i)}-\tilde{x}^{(j}$ for ease of notation.
\begin{align*}
     E^{cable}_{elast}(\lambda e_{ij} + (1-\lambda) \tilde{e_{ij}}) &= \frac{k}{2 \el^2} \bigg( \lVert \lambda p + ( 1-\lambda ) \tilde{p} \rVert -\el\bigg)^2
\end{align*}
\textbf{Question: Does $\ece (\nabla \el + (1-\lambda) \tilde{\el}$ notation even make sense? }

Recall that if the expression inside the norm is less than $\el$, we get $0$. In either case, we can use the triangle inequality to get
\begin{align*}    
     &E^{cable}_{elast}(\lambda e_{ij} + (1-\lambda) \tilde{e_{ij}})\\ &=
      \frac{k}{2 \el^2} \bigg( \lambda \p +( 1-\lambda ) \pt -\el\bigg)^2 \\ 
     &=\frac{k}{2 \el^2} \bigg( \lambda^2 \p^2 + 2 \lambda (1-\lambda) \p \pt -2 \lambda \p \el -2(1-\lambda) \pt \el + (1-\lambda)^2 \pt^2 + \el^2 \bigg) \\
     &<\frac{k}{2 \el^2} \bigg( \lambda \p^2 + 2 (1-\lambda) \p \pt -2 \lambda \p \el -2(1-\lambda) \pt \el + (1-\lambda) \pt^2 + \el^2 \bigg) \\ & \text{because }\lambda, (1-\lambda) < 1 \text{ and the expressions we alter are positive.} \\
     &= \frac{k}{2 \el^2} \bigg(\lambda \p^2 + 2 (1-\lambda) \p \pt - 2 \lambda \p \el -2\pt \el +2 \lambda \pt \el + \pt^2 - \lambda \pt^2 + \el^2 \bigg) \\
     &< \frac{k}{2 \el^2} \bigg(\lambda \p^2 - 2 \lambda \p \el -2\pt \el +2 \lambda \pt \el + \pt^2 - \lambda \pt^2 + \el^2 \bigg)\\
     &= \frac{k}{2 \el^2} \bigg( \lambda \p^2 - 2 \lambda \p \el + \lambda \el^2 + \pt^2 - 2 \pt \el + \el^2 - \lambda \pt^2 + 2\lambda \pt \el - \lambda \el^2  \bigg)\\
     &= \frac{k}{2 \el^2} \bigg( \lambda \big(\p^2 - 2 \p \el + \el^2\big) + (1-\lambda) \big(\pt^2 -2 \pt \el + \el^2\big) \bigg)\\
     &= \lambda \frac{k}{2 \el^2}\bigg( \p - \el \bigg)^2 + ( 1-\lambda )\frac{k}{2 \el^2}\bigg( ( \pt -\el ) \bigg)^2 
     \\&= \lambda E^{cable}_{elast}(e_{ij}) + (1-\lambda) E^{cable}_{elast}(\Tilde{e_{ij}})&&  \square
\end{align*}

We see that the function for a given node is convex. Therefore, $ E(X)$ is also convex as it's a sum of convex functions. The fact that this energy expression is convex means that Quasi-Newton methods are a good candidate. As our function is not $C^2$, we Newton's Method is not an option. 

\textbf{QUESTION:} Do we have guaranteed convergence for BFGS when the function is convex but not $C^2$?.

\subsection{Neccesary and sufficient optimality conditions}
As we have a convex function that is differentiable, the neccesary and sufficient optimality condition for a solution $X^*$ is simply \begin{equation}
    \nabla E(X^*) = 0
\end{equation}
This will be a global minimizer, again due to convexity.