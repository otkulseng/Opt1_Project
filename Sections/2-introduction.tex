\section{Introduction}
\subsection{Modeling of the structures}
Tensegrity structures consist of bars and cables that are connected with joints. We will model the structures as a \textbf{(DIRECTED?)} graph $\mathcal{G} = (\mathcal{V},\mathcal{E})$, where $\mathcal{V} = \{1,...,N\}$ is a set of vertices, and $\mathcal{E} \subset \mathcal{V} \times \mathcal{V}$ is a set of edges. The vertices naturally represent the joints of the structure, and an edge $e_{ij} = (i,j)$ with $i < j$ indicates that the joints $i$ and $j$ are connected through either a cable or a bar.

The position of a node $i$ is given by $x^{(i)} = (x_1^{(i)},x_2^{(i)},x_3^{(i)})$. Additionally, we will collect the position of all nodes in a vector $X = (x^{(1)},...,x^{(N)} \in \mathbb{R}^{3N}$

The goal is to determine the position $X$ of all the nodes. We rely on the fundamental physical principle that the structure will attain a stable resting position $X^*$ only when the total potential energy of the system has a local or global minimum. This naturally gives us an optimization problem.

We will assume that all bars are made of the same material, and have identical thickness and cross section. However they can differ in length. A bar $e_{ij}$ has a resting length $\ell_{ij}>0$, where the internal elastic energy is $0$. If the bar is stretched or compressed to a new length $L(e_{ij})=\lVert x^{(i)} - x^{(j)}\rVert$, we will model the energy using a quadratic model


\begin{equation}
    \ebe = \frac{c}{2\el^2}(L\e - \el)^2 = \frac{c}{2 \el^2}(\xnorm - \el)^2
    \label{barElast}
\end{equation}
where the parameter $c > 0$ depends on the material and cross section of the bar. We also consider the potential energy of the bar, as it has a considerable mass.

\begin{equation}
    \ebg = \frac{\rho g \el}{2}(x_3^{(i)}+x_3^{(j)})
    \label{barGrav}
\end{equation}

Cables are modeled similarly, we only permit varying length. A cable has a resting length $\el > 0$, where the internal elastic energy is $0$. Compression of a cable yields no energy, but stretching will be modeled similarly to a bar. This gives us

\begin{equation}
\ece = \begin{cases}
    \frac{k}{2\el^2}(\xnorm-\el)^2 & \text{if} \quad \xnorm >\el\\
    0 & \text{if} \quad \xnorm \leq \el
    \end{cases}
    \label{cableElast}
\end{equation}
where $k > 0$ is a material parameter, $\rho$ is the mass density, and $g$ is the gravitational acceleration. Additionally, we consider the weight of the cables negligible compared to the weight of the bars. That is:
\begin{equation}
    E^{cable}_{bar}\e = 0
    \label{cableGrav}
\end{equation}

We will also model external loads for a given node. If node $i$ is loaded with mass $m_i \geq 0$, this will result in the total external energy
\begin{equation}
    \ee = \sum_{i=1}^{N} m_i g x_3^{(i)}
    \label{externalEnergy}
\end{equation}

We can express the total energy of the structure as \begin{equation}
    E(X) = \sumset{B}(\ebe + \ebg) + \sumset{C} \ece + \ee
    \label{totalEnergy}
\end{equation} where $\mathcal{B}, \mathcal{C} \subset \mathcal{E}$ are the sets of bars and cables in the structure. This function is continious, the only potential problem is the piecewise function $\ece$ at the point where $L\e = \xnorm = \el$, but we see that it evaluates to $0$, so this term is also continious.

Note that minimizing \eqref{totalEnergy} might not admit a solution, as the energy can be unbounded from below by letting all $z$-coordinates of the nodes tend to $-\infty$. We propose two constraints to prevent this issue.

\subsection{Fixing the position of a set of nodes}
The first constraint is fixing some of the nodes such that
\begin{equation}
    x^{(i)} = p^{(i)} \qquad \text{for } i = 1,...,M
\end{equation} for some fixed $p^{(i)} \in \mathbb{R}^3$, and $1\leq M < N$

This constraint is nice because we still have a free optimization problem, where we have replaced some $x^{(i)}$ by $p^{(i)}$. The dimension of $X$ is now $3(N-M)$

\textbf{Kommentar: Føringen under føles ikke mega-presis, kan sikkert forbedres}
If we assume that we have a connected graph $\mathcal{G}$, meaning the structure is connected, then simply fixing one node, for example $x^{(1)}$ in this way is enough to give us coercivity. We can show this by noticing that if we send for example $x^{(2)}$ to $-\infty$ and keep all other $x_i \in X \setminus x^{(2)}$ bounded we get
\begin{equation}
  \xinf \ee = -\infty
\quad\text{and additionally}\quad
\xinf \sumset{B} \ebg = -\infty \quad \text{if $e_{ij}$ is a bar}
\label{minusinf}
\end{equation}
However, we also get
\begin{equation}
     \xinf \sumset{C}\quad \ece = \infty \quad \text{or}\quad \xinf \sumset{B }\ebe = \infty \quad
     \label{plusinf1}
\end{equation} (depending on whether the node is a cable or bar)
due to the fact that both these sums contain the term \begin{equation}
 \xinf \frac{k}{2 \ell_{p2}^2}(\lVert x^{(p)} - x^{(2)} \rVert-\ell_{p2})^2 = \infty \quad \text{or} \quad \xinf \frac{c}{2 \ell_{p2}^2}(\lVert x^{(p)} - x^{(2)} \rVert-\ell_{p2})^2 = \infty
 \label{plusinf2}
\end{equation}
and it's clear that the terms in \eqref{minusinf} will be dominated one of the terms in \eqref{plusinf1} because they contain one of the terms in \eqref{plusinf2} with the exponent of $2$. Hence, the total energy function \eqref{totalEnergy} is coercive. If we send multiple terms to $\pm \infty$ we will get similar results, and the choice of fixing the node $x^{(i)}$ and sending $x^{(2)}$ was arbitrary, so this shows that we have coercivity in all cases of $\xinf$. We have already shown that the function is continious, therefore it's also lower semi continious, and this implies that the minimisation problem admits a solution. $\square$

Note that we indeed needed to assume that the graph was connected. Without this assumption, we would have to fix one node from each of the disconnected graphs.

\subsection{Imposing positive z-values of the nodes}
The second constraint models a self-supported free standing structure, with the only condition being that it's above ground:
\begin{equation}
    x_3^{(i)} \geq 0 \quad \forall \quad i = 1,...,N
\end{equation}

Note that coerciveness is not as straight forward in this case. If we let $x_1^{(i)}, x_1^{(j)} \to \infty$ and keeping all other variables constant we see that the distance between the nodes do not change, and therefore the energy does not tend to $\infty$. This can be fixed without a loss of generality by fixing the $x_1$ and $x_2$-position of a given node: $x^{(i)} = (p_1,p_2,x^{(i)}_3)$.

With this setup, showing coerciveness is very similar to the previous condition. Note that this is indeed an actual constrained optimization problem. 

\textbf{Kommentar: Er dette godt nok begrunnet?}

\textbf{Og hva er forskjellen på dette og P11, referer det ikke til samme problem?}
