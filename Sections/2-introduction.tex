\section{Introduction}
\subsection{Modeling of the structures}
Tensegrity structures consist of bars and cables that are connected with joints. We will model the structures as a directed graph $\mathcal{G} = (\mathcal{V},\mathcal{E})$, where $\mathcal{V} = \{1,...,N\}$ is a set of vertices, and $\mathcal{E} \subset \mathcal{V} \times \mathcal{V}$ is a set of edges. The vertices naturally represent the joints of the structure, and an edge $e_{ij} = (i,j)$ with $i < j$ indicates that the joints $i$ and $j$ are connected through either a cable or a bar.

The position of a node $i$ is given by $x^{(i)} = (x_1^{(i)},x_2^{(i)},x_3^{(i)})$. Additionally, we will collect the position of all nodes in a vector $X = (x^{(1)},...,x^{(N)} \in \mathbb{R}^{3N}$

The goal is to determine the position $X$ of all the nodes. We rely on the fundamental physical principle that the structure will attain a stable resting position $X^*$ only when the total potential energy of the system has a local or global minimum. This naturally gives us an optimization problem.

We will assume that all bars are made of the same material, and have identical thickness and cross section. However, they can differ in length. A bar $e_{ij}$ has a resting length $\ell_{ij}>0$, where the internal elastic energy is $0$. If the bar is stretched or compressed to a new length $L(e_{ij})=\lVert x^{(i)} - x^{(j)}\rVert$, we will model the energy using a quadratic model
\begin{equation}
    \ebe = \frac{c}{2\el^2}(L\e - \el)^2 = \frac{c}{2 \el^2}(\xnorm - \el)^2
    \label{barElast}
\end{equation}
where the parameter $c > 0$ depends on the material and cross section of the bar. We also consider the potential energy of the bar, as it has a considerable mass. With $\rho$ denoting the line density of the bar and $g$ the gravitational acceleration, we have that
\begin{equation}
    \ebg = \frac{\rho g \el}{2}(x_3^{(i)}+x_3^{(j)})
    \label{barGrav}
\end{equation}

Cables are modeled similarly, we only permit varying length. A cable has a resting length $\el > 0$, where the internal elastic energy is $0$. Compression of a cable yields no energy, but stretching will be modeled similarly to a bar. This gives us

\begin{equation}
\ece = \begin{cases}
    \frac{k}{2\el^2}(\xnorm-\el)^2 & \text{if} \quad \xnorm >\el\\
    0 & \text{if} \quad \xnorm \leq \el
    \end{cases}
    \label{cableElast}
\end{equation}
where $k > 0$ is a material parameter. Additionally, we consider the weight of the cables negligible compared to the weight of the bars.

We will also model external loads for a given node. If node $i$ is loaded with mass $m_i \geq 0$, this will result in the total external energy
\begin{equation}
    \ee = \sum_{i=1}^{N} m_i g x_3^{(i)}
    \label{externalEnergy}
\end{equation}

Hence, we can express the total energy of the structure as \begin{equation}
    E(X) = \sumset{B}(\ebe + \ebg) + \sumset{C} \ece + \ee
    \label{totalEnergy}
\end{equation} where $\mathcal{B}, \mathcal{C} \subset \mathcal{E}$ are the sets of bars and cables in the structure. This function is continuous, the only potential problem is the piecewise continuous function $\ece$ at the point where $L\e = \xnorm = \el$. However, we see that it evaluates to $0$, so this term is also continuous.

Note that minimizing \eqref{totalEnergy} might not admit a solution, as the energy can be unbounded from below by letting all $z$-coordinates of the nodes tend to $-\infty$. We propose two solutions to this issue.

\subsection{Fixing the position of a set of nodes}
The first option is fixing some of the nodes such that
\begin{equation}
    x^{(i)} = p^{(i)} \qquad \text{for } i = 1,...,M
    \label{fixednode}
\end{equation} for some fixed $p^{(i)} \in \mathbb{R}^3$, and $1\leq M < N$. This constraint is convenient because we still have a free optimization problem, where we have replaced some $x^{(i)}$ by $p^{(i)}$. The dimension of $X$ is now $3(N-M)$

\textbf{Theorem: If the graph $\mathcal{G}$ is connected, the objective function \eqref{totalEnergy} with the constraint \eqref{fixednode} admits a solution.}

Proof:
We start by showing coercivity:
Using that $\mathcal{G}$ is connected, the entire structure is connected through either cables or bars. This means that for any free node in the structure, there exists some set of nodes $\mathcal{S}$ that defines a path to a fixed node $p^{(i)}$. 

If we consider any possible combination of $x^{(i)}_1 \to \pm \infty,x^{(i)}_2 \to \pm \infty,x^{(i)}_3 \to \pm \infty$, the average length of edges in $\mathcal{S}$ will also tend to $\infty$. For cables we will only consider the case when $\xnorm > \el$ as the energy will be equal. The elastic energy in $E(X)$ will be given by
\begin{equation} 
\label{plusinf1}
\begin{aligned}
     &\xinf \sumset{C}\ece =\xinf \sumset{C} \frac{k}{2 \el ^2}(\lVert x^{(i)} - x^{(j)} \rVert-\el)^2 = \infty \\
     &\xinf \sumset{B }\ebe = 
      \sumset{B} \frac{c}{2 \el ^2}(\lVert x^{(i)} - x^{(j)} \rVert-\el)^2 = \infty
\end{aligned} 
\end{equation}

The fact that we allow $x^{(i)}_3 \to -\infty$ could potentially result in
\begin{equation}
  \xinf \ee = -\infty
\quad\text{and additionally}\quad
\xinf \sumset{B} \ebg = -\infty \quad \text{if $e_{ij}$ is a bar}
\label{minusinf}
\end{equation} 
However, it's clear that the terms in \eqref{minusinf} will be dominated by one of the terms in \eqref{plusinf1} because they contain quadratic terms. Hence, the total energy function \eqref{totalEnergy} is coercive.

We have already shown that the function is continuous, therefore it's also lower semi-continuous and this implies that the minimisation problem admits a solution. \hfill $\square$

If we have a disconnected graph, we would have to split the graph into subgraphs that are connected, and fix at least one node in every connected subgraph. We will not consider these situations in this paper, so we will not prove this.

\subsection{Imposing positive z-values of the nodes}
The second constraint models is a self-supported free standing structure, with the only condition being that it's above ground:
\begin{equation}
    x_3^{(i)} \geq 0 \quad \forall \quad i = 1,...,N
    \label{z_positive}
\end{equation}
Note that coerciveness is not as immediate in this case. If we simultaneously move all the nodes horizontally in any direction, we see that the distance between the nodes do not change, and thus the energy is constant. This issue can be solved without a loss of generality by fixing the $x_1$ and $x_2$-position of a given node: $x^{(i)} = (p_1,p_2,x^{(i)}_3)$. This simply disallows moving the entire structure horizontally.

\textbf{Theorem: If the graph $\mathcal{G}$ is connected, the objective function \eqref{totalEnergy} with the constraint \eqref{z_positive} admits a solution.}

With this setup, coerciveness mostly follows from the proof in the theorem above. Note that we do not allow $x_3 \to -\infty$ because of the constraint \eqref{z_positive}. Additionally, note that the energy from $\ece$ and $\ebe$ does not tend to $\infty$ when we increase $z$ simultaneously for all nodes. However, in this case the external force and bar weight will increase, and thus the total energy function is coercive. \hfill $\square$

Note that this restriction indeed creates a constrained optimization problem, unlike the constraint \eqref{fixednode} where we had a free optimization problem in a lower dimension.