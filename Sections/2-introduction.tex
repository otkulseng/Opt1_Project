\section{Introduction}
\subsection{Modeling of the structures}
Tensegrity structures consist of bars and cables that are connected with joints. We will model the structures as a directed graph $\mathcal{G} = (\mathcal{V},\mathcal{E})$, where $\mathcal{V} = \{1,...,N\}$ is a set of vertices, and $\mathcal{E} \subset \mathcal{V} \times \mathcal{V}$ is a set of edges. The vertices naturally represent the joints of the structure, and an edge $e_{ij} = (i,j)$ with $i < j$ indicates that the joints $i$ and $j$ are connected through either a cable or a bar.

The position of a node $i$ is given by $x^{(i)} = (x_1^{(i)},x_2^{(i)},x_3^{(i)})$. Additionally, we will collect the position of all nodes in a vector $X = (x^{(1)},...,x^{(N)} \in \mathbb{R}^{3N}$

The goal is to determine the position $X$ of all the nodes. We rely on the fundamental physical principle that the structure will attain a stable resting position $X^*$ only when the total potential energy of the system has a local or global minimum. This naturally gives us an optimization problem.

We will assume that all bars are made of the same material, and have identical thickness and cross section. However they can differ in length. A bar $e_{ij}$ has a resting length $\ell_{ij}>0$, where the internal elastic energy is $0$. If the bar is stretched or compressed to a new length $L(e_{ij})=\lVert x^{(i)} - x^{(j)}\rVert$, we will model the energy using a quadratic model


\begin{equation}
    \ebe = \frac{c}{2\el^2}(L\e - \el)^2 = \frac{c}{2 \el^2}(\xnorm - \el)^2
    \label{barElast}
\end{equation}
where the parameter $c > 0$ depends on the material and cross section of the bar. We also consider the potential energy of the bar, as it has a considerable mass.

\begin{equation}
    \ebg = \frac{\rho g \el}{2}(x_3^{(i)}+x_3^{(j)})
    \label{barGrav}
\end{equation}

Cables are modeled similarly, we only permit varying length. A cable has a resting length $\el > 0$, where the internal elastic energy is $0$. Compression of a cable yields no energy, but stretching will be modeled similarly to a bar. This gives us

\begin{equation}
\ece = \begin{cases}
    \frac{k}{2\el^2}(\xnorm-\el)^2 & \text{if} \quad \xnorm >\el\\
    0 & \text{if} \quad \xnorm \leq \el
    \end{cases}
    \label{cableElast}
\end{equation}
where $k > 0$ is a material parameter, $\rho$ is the mass density, and $g$ is the gravitational acceleration. Additionally, we consider the weight of the cables negligible compared to the weight of the bars. That is:
\begin{equation}
    E^{cable}_{bar}\e = 0
    \label{cableGrav}
\end{equation}

We will also model external loads for a given node. If node $i$ is loaded with mass $m_i \geq 0$, this will result in the total external energy
\begin{equation}
    \ee = \sum_{i=1}^{N} m_i g x_3^{(i)}
    \label{externalEnergy}
\end{equation}

We can express the total energy of the structure as \begin{equation}
    E(X) = \sumset{B}(\ebe + \ebg) + \sumset{C} \ece + \ee
    \label{totalEnergy}
\end{equation} where $\mathcal{B}, \mathcal{C} \subset \mathcal{E}$ are the sets of bars and cables in the structure. This function is continious, the only potential problem is the piecewise function $\ece$ at the point where $L\e = \xnorm = \el$, but we see that it evaluates to $0$, so this term is also continious.

Note that minimizing \eqref{totalEnergy} might not admit a solution, as the energy can be unbounded from below by letting all $z$-coordinates of the nodes tend to $-\infty$. We propose two constraints to prevent this issue.

\subsection{Fixing the position of a set of nodes}
The first constraint is fixing some of the nodes such that
\begin{equation}
    x^{(i)} = p^{(i)} \qquad \text{for } i = 1,...,M
    \label{fixednode}
\end{equation} for some fixed $p^{(i)} \in \mathbb{R}^3$, and $1\leq M < N$. This constraint is convenient because we still have a free optimization problem, where we have replaced some $x^{(i)}$ by $p^{(i)}$. The dimension of $X$ is now $3(N-M)$

\textbf{formuler som teorem}

If we assume that we have a connected graph $\mathcal{G}$, meaning the structure is connected, then simply fixing one node, for example $x^{(1)} = p^{(1)}$ in this way is enough to give us coercivity.

\textbf{kommentar: teksten under føles ikke veldig presis.}
 \textbf{Average distanse mellom noder går mt uendelig, og da også energien}
We can see this by using the fact that our graph is connected - this means that our fixed node is connected to a cable, bar, or both. For any node $x^{(i)} \neq p^{(1)}$, it will either be directly connected to $p^{(1)}$, or there will exist a path to $p^{(1)}$ through some set of nodes in $\mathcal{E}$. If we send $\lVert X \rVert \to \infty$ through any possible combination of $x^{(i)}_1 \to \pm \infty,x^{(i)}_2 \to \pm \infty,x^{(i)}_3 \to \infty$ then we have $$\xinf E(X)\to \infty$$ because at least one of the cables/bars on the path between $x^{(i)}$ and $p^{(1)}$ will be stretched towards length $\infty$, and therefore it's elastic energy will tend to $\infty$. 

If we additionally allow $x^{(i)}_3 \to -\infty$ we could potentially have some combination of sending nodes to $\pm \infty$ that give us
\begin{equation}
  \xinf \ee = -\infty
\quad\text{and additionally}\quad
\xinf \sumset{B} \ebg = -\infty \quad \text{if $e_{ij}$ is a bar}
\label{minusinf}
\end{equation} 
However, we also get
\begin{equation} 
\label{plusinf1}
\begin{aligned}
     &\xinf \sumset{C}\quad \ece = \infty \quad \text{and/or}\quad \xinf \sumset{B }\ebe = \infty \quad\\ 
     %&\text{depending on whether the fixed node is connected to a cable or bar or both}
\end{aligned} 
\end{equation}
depending on whether the fixed node is connected to a cable or bar or both. This is due to the fact that both these sums contain the term \begin{equation}
 \xinf \frac{k}{2 \ell_{p2}^2}(\lVert x^{(p)} - x^{(2)} \rVert-\ell_{p2})^2 = \infty \quad \text{and/or} \quad \xinf \frac{c}{2 \ell_{p2}^2}(\lVert x^{(p)} - x^{(2)} \rVert-\ell_{p2})^2 = \infty
 \label{plusinf2}
\end{equation}
It's clear that the terms in \eqref{minusinf} will be dominated one of the terms in \eqref{plusinf1} because they contain one of the quadratic terms in \eqref{plusinf2}. Hence, the total energy function \eqref{totalEnergy} is coercive. This concludes that for every possible combination we have $\xinf E(X) = \infty$, so we have shown coercivity. We have already shown that the function is continious, therefore it's also lower semi continious, and this implies that the minimisation problem admits a solution. $\square$

If we have a disconnected graph, we would have to fix at least one node in every connected subgraph contained in the disconnected graph. \textbf{but we will not proove this}

\subsection{Imposing positive z-values of the nodes}
The second constraint models a self-supported free standing structure, with the only condition being that it's above ground:
\begin{equation}
    x_3^{(i)} \geq 0 \quad \forall \quad i = 1,...,N
\end{equation}
\textbf{Rydd opp i dette}

Note that coerciveness is not as immediate in this case. If we simultaniously move all the nodes horizontally in any direction, we see that the distance between the nodes do not change, and thus the energy is constant. This can be fixed without a loss of generality by fixing the $x_1$ and $x_2$-position of a given node: $x^{(i)} = (p_1,p_2,x^{(i)}_3)$. This simply disallows moving the entire structure horisontally.

With this setup, showing coerciveness is very similar to the previous section. In fact it's even easier because we can no longer let $x^{(i)}_3 \to -\infty$ which was the only potential issue.

Note that this restriction indeed creates a constrained optimization problem, unlike the full fixation of a node \eqref{fixednode} where we still had a free optimization problem in a lower dimension.

\textbf{Kommentar: Er dette godt nok begrunnet? Føltes litt waste å kjøre hele coercive-argumentet på nytt..}

