\section{Numerical}
The code used to solve the optimization problems and to generate the plots used can be found in \href{https://github.com/otkulseng/Opt1_Project}{this} Github repository.

There are two major parts of the code. 

One part is a completely generic implementation of the BFGS method in the \href{https://github.com/otkulseng/Opt1_Project/blob/main/Kode/algoritmer.py}{algoritmer.py} file. This implementation follows closely that of algorithm 6.1 in \cite{NW}, the only difference being the choice of steplength. Our implementation use a linesearch method to find a steplength that satisfies the \emph{strong} Wolfe conditions, rather than the regular ones used in the book. The linesearch metjhod

The BFGS iterations stop either when some predetermined maximum iterations has been reached, or when the norm of the gradient is below a threshold of $\num{1e-10}$





\subsection{Cable nets}
For the first experiment we will consider $4$ free and $4$ fixed nodes along with following parameters:
\begin{equation*}
\begin{gathered}
    4 \text{ fixed nodes } p^{(1)} = (5,5,0),\; p^{(2)} = (-5,5,0),\; p^{(3)} = (-5,-5,0),\; p^{(4)} = (5,-5,0) \\
    \mathcal{E} = \{(1,5),\;(2,6),\;(3,7),\;(4,8),\; (5,6),\; (6,7),\; (7,8),\; (8,5) \}\\
    k=3,\quad \el = 3 \quad \forall \quad (i,j) \in \mathcal{E}, \quad m_i g = \frac{1}{6}, \quad i= 5,6,7,8 
\end{gathered}
\end{equation*}

This problem has a analytical solution for the free nodes:
\begin{equation*}
    \begin{gathered}
    x^{(5)} = (2,2,-\frac{3}{2}),\;x^{(6)} = (-2,2,-\frac{3}{2}),\;x^{(5)} = (-2,-2,-\frac{3}{2}),\;x^{(5)} = (2,-2,-\frac{3}{2})
    \end{gathered}
\end{equation*}


\begin{figure}
    \centering
    \includegraphics[width=0.6\columnwidth]{Bilder/P25.pdf}
    \caption{Cable net structure}
    \label{P25}
\end{figure}
We see from \eqref{P25} that we indeed reach this configuration of nodes. 

\textbf{Noe om at den ender opp i samme løsning uansett startkonfigurasjon?}

\subsection{Tensegrity domes}
We now consider bars as well, and will use the $4$ fixed nodes and the following parameters:

\begin{equation*}
    \begin{gathered}
    p^{(1)} = (1,1,0),\; p^{(2)} = (-1,1,0),\; p^{(3)} = (-1,-1,0),\; p^{(4)} = (1,-1,0)\\
    \ell_{15} = \ell_{26} = \ell_{37} = \ell_{48} = 10, \qquad \ell_{18} = \ell_{25} = \ell_{36} = \ell_{47} = 8, \qquad \ell_{56} = \ell_{67} = \ell_{78} = \ell_{58} = 1\\
    c=1, \quad k= 0.1, \quad g \rho = 0,\quad m_i g = 0, \quad i = 5,6,7,8
    \end{gathered}
\end{equation*}

\begin{figure}
    \centering
    \includegraphics[width=0.6\columnwidth]{Bilder/P69.pdf}
    \caption{Tensegrity dome}
    \label{P69}
\end{figure}
Again, we have an analytical solution to this problem, namely 
\begin{equation*}
    \begin{gathered}
    x^{(5)} = (-s,0,t),x^{(6)} = (0,-s,t),x^{(7)} = (s,0,t),x^{(8)} = (0,s,t),  \text{with}\; s \approx 0.70970, \; t \approx 9.54287
    \end{gathered}
\end{equation*}

\begin{figure}
    \centering
    \includegraphics[width=0.6\columnwidth]{Bilder/FREESTANDING.pdf}
    \caption{Free standing}
    \label{fig:freestanding}
\end{figure}

\begin{figure}
    \centering
\includegraphics[width=0.6\columnwidth]{Bilder/2FREESTANDING.pdf}
    \caption{2 free standing stacked}
    \label{fig:2freestanding}
\end{figure}


